\documentclass[a4paper,12pt]{article}

\usepackage[dutch]{babel}
\usepackage[T1]{fontenc}
\usepackage[dutch]{isodate}
\usepackage{parskip}
\usepackage{todonotes}

\begin{document}

\title{Beeldverwerking: Practicum noise removal}
\author{Frank Erens \and Randy Thiemann}
\date{\isodate\today}
\maketitle

\section{Inleiding}

Het doel van dit project was de drie varianten van het algoritme zoals gegeven
in de paper van Hirani et al., 1996 te implementeren in MatLab met een
met een geschikte grafische interface.

In dit verslag beschrijven we kort de stappen die we hiertoe hebben ondernomen
en de problemen die we tegen zijn gekomen.

\section{Inhoud archief}

Het ingestuurde archief bevat, naast dit verslag, alle broncode die nodig is om
het project in MatLab te draaien. 

In de map \texttt{examples/} zijn een aantal voorbeeldafbeeldingen te vinden
waar het programma op kan getest worden. Voor iedere afbeelding zijn er ook
twee maskers beschikbaar, \'e\'en binair (\texttt{\_mask}) \'e\'en zacht 
(\texttt{\_maskf}), als ook een voorbeeld van de uitvoer na het algoritme 20
keer gedraaid te hebben (\texttt{\_out}).

\section{Ge\"implementeerde functionaliteit}

Het programma laat toe om krassen op afbeeldingen te verwijderen via een
algoritme zoals beschreven in de paper van Hirani et al., 1996. In de paper
worden drie varianten van het algoritme beschreven, gelabeld A1 tot A3. 

\todo{}
XXXX Alle drie geimplementeerd XXXX

Om de grafische interface te voorzien werd er gebruik gemaakt van de ingebouwde
gui design tool van MatLab. Deze interface laat de gebruiker toe om een
afbeelding in te laden en uit te schrijven, en laat de gebruiker ook toe de
nodige parameters in te stellen.

\section{Gebruik}

Om het programma te starten, open MatLab en navigeer naar de map van het
project en typ vervolgens \texttt{Project} in het commandovenster.

Dit start de grafische interface. Deze bestaat uit een gebied waar de
afbeelding weergegeven wordt, als ook een aantal knoppen om het algoritme te
bedienen.

Als eerste stap dient er een afbeelding geladen te worden 
(\textbf{File} $\to$ \textbf{Load Image}). Deze afbeelding wordt dan
weergegeven in het veld.

Nadat de afbeelding is ingeladen, kan het algoritme geconfigureerd worden met
behulp van een aantal knoppen aan de linkerkant. Deze knoppen dienen in
volgorde afgelopen te worden. Wanneer een stap klaar is, zal de knop groen
oplichten. Tijdens het proces kan ten allen tijden een stap opnieuw worden
uitgevoerd, maar dan dienen alle daaropvolgende stappen ook opnieuw uitgevoerd
te worden.

De eerste parameter voor het algoritme is een masker. Dit is een afbeelding die
even groot is als de oorspronkelijke afbeelding, die wit is, met het gebied van
de kras zwart gemarkeerd. Het is ook mogelijk om een masker te gebruiken met
zachte randen hier. Dit masker dient op voorhand gemaakt te worden in een
extern programma, bijvoorbeeld Photoshop.

Om het masker in te laden, klik \textbf{Load Noise Mask}. Een bestandsdialoog
zal verschijnen. Kies het correcte masker voor de afbeelding. Indien dit
correct is gedaan, zal de knop groen oplichten en het masker over de afbeelding
getekend worden. 

De volgende stap is het selecteren van het gebied waarin de reparatie moet
plaatsvinden. Klik hiervoor op de knop \textbf{Mark Repair}. Trek vervolgens
een rechthoek over het gebied met de kras. Indien dit correct is gedaan, zal de
knop wederom groen oplichten.

Ten slotte moet er ook een gebied worden aangeduid dat krasvrij is. Dit gebied
moet, om de beste resultaten te leveren, visueel overeenkomstig zijn met het
gebied van de kras in. Klik op de knop \textbf{Mark Sample}, en duid vervolgens
de linkerbovenhoek aan van het krasvrije gebied. Het programma zal automatisch
een gebied ter grote van het gemarkeerde krasgebied nemen. Als deze stap ook
correct is uitgevoerd, zal deze knop ook groen oplichten.

Nadat alle parameters ingegeven zijn, kan het algoritme gestart worden.
Hierbij kan de keuze gemaakt worden tussen de drie ge\"implementeerde
algoritmen. Er zijn twee knoppen beschikbaar, de eerste, 
\textbf{Run Algorithm 1/2} draait het eerste dan wel het tweede algoritme
naargelang het gebruikte masker monochroom of met zachte randen was. De tweede
knop, \textbf{Run Algorithm 3}, draait het derde algoritme, wat betere
resultaten kan leveren bij afbeeldingen waar er een groot contrast in
intensiteit is tussen de repair- en samplegebieden.

Bij het drukken op de knop zal \'e\'en iteratie van het algoritme uitgevoerd
worden. De gebruiker kan zoveel iteraties uitvoeren als gewenst zijn. Na 10 tot
20 iteraties zou de kras zo goed als hersteld moeten zijn.

Voor afbeeldingen met meer dan een kras, herhaal deze stappen zo veel als
nodig. Het is niet noodzakelijk om verschillende maskers te gebruiken voor
verschillende krassen, het programma detecteert automatisch de juiste kras via
de selectie van de gebruiker.

Wanneer alle krassen naar wens verwijderd zijn, kan de resulterende afbeelding
terug weggeschreven worden (\textbf{File} $\to$ \textbf{Save Image}).

\section{Conclusie}
\todo{}
XXXX Problemen met implementatie algoritmes beschrijven.

Bij het ontwerpen van de interface hebben we vooral problemen gehad met de
bizarre manier waarop MatLab werkt, deze problemen waren echter makkelijk op te
lossen met wat raadplegen van de documentatie.

Op deze kleine problemen na is dit project redelijk vlot ge\"implementeerd.


  
\end{document}

% vim: spelllang=nl,en
